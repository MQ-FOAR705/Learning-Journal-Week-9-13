\documentclass{article}
\usepackage[utf8]{inputenc}
\usepackage{hyperref}
\usepackage{graphicx}

\hypersetup{
    colorlinks=true,
    linkcolor=blue,
    filecolor=magenta,      
    urlcolor=cyan,
    pdftitle={Sharelatex Example},
    bookmarks=true,
    pdfpagemode=FullScreen,
    }

\title{Learning Journal: Week 9-13}
\author{Sophie Avard}
\date{October 2019}

\begin{document}

\maketitle

\section{Introduction}



\section{User Story 4}
\subsection{Intention}
Add a tool to Zotero to automatically add files library.

\subsection{Action}
\begin{enumerate}
    \item Go to Chrome Web Store and search for Zotero 
    \item Click the 'add to Chrome' button 
    \item Login via the connector using my Zotero account
    \item Click pdf button on top right of screen to save metadata and pdf straight into Zotero
\end{enumerate}
\subsection{Error}
No error but it is hard to locate PDF's when downloaded as the file name is often hard to read. 

\section{User Sotry 4}
\subsection{Intention}
Add a tool to Zotero to automatically change the name of the PDF's.
\subsection{Action}
\begin{enumerate}
    \item Install ZotFile: \href{http://zotfile.com/#features}{http://zotfile.com/features}  
    \item In Zotero go to tools - Add-ons - Tools for all Add-ons - Install Add-on from file - selection the downloaded .xpi file
    \item Change the source folder for attaching new files: Tools - ZotFile Preferences - set to User/sophieavard/Downloads
    \item In Renaming Rule tab: Click 'use Zotero to rename' and format item types as: \verb|author lastname_year_title| (see image below)
\end{enumerate}
   \includegraphics[width=12cm]{zotfile.png}
\subsection{Error}
NA. 
\section{User Story 2}
\subsection{Intention}
Retrieve metadata from Zotero in a report format. 
\subsection{Action}
\begin{enumerate}
\item Select items to be included in metadata report
\item Right click - click 'generate report for items' 
\end{enumerate}
\subsection{Error}
No error but report isn't very flexible. Doesn't allow me to edit the metadata. 

\section{User Story 2}
\subsection{Intention}
Add a tool to help make metadata report more flexible. 
\subsection{Action}
\begin{enumerate}
    \item Download Report Customiser for Zotero: \href{https://github.com/retorquere/zotero-report-customizer}{Zotero Report Customiser}
    \item In Zotero click toools - Add-ons - Extensions
    \item Click on the gear in the top-right and choose 'install add-on from file'
    \item Choose the .xpi that was just downloaded and click 'Install'
    \item Re-start Zotero
\end{enumerate}
Produce report:
\begin{enumerate}
    \item Select items to be included in report
    \item right click - click 'generate report for items' 
    \item Click edit button in top left corner 
    \item Edit metadata report (See image below)
\end{enumerate}
\includegraphics[width=12cm]{metadata.png}

\section{User Story 8}
\subsection{Intention}
Produce a table/graph to help draw connections and relationships between items. Download ZotNet tool to allow me to see relationships and calculate basic statistics. \\
\textbf{ZotNet}: \href{https://www.exitsantacruz.com/zotnet/zotnet.php}{https://www.exitsantacruz.com/zotnet/zotnet.php}
\subsection{Action}
\begin{enumerate}
    \item Download NetDraw to open Zotnet analysis: \href{https://sites.google.com/site/netdrawsoftware/home}{NetDraw}
    \item Click 'Installation Package'
    \item Follow the prompts on the screen 
\end{enumerate}
\subsection{Error}
Download didn't work.

\section{User Story 8}
\section{Intention}
Download and install UCINET which will automatically install NetDraw.
\section{Action}
\begin{enumerate}
    \item Go to: \href{https://sites.google.com/site/netdrawsoftware/home}{NetDraw}
    \item Click UCINET
    \item Download 32-bit Installation package 
    \item Read installation notes at bottom
\end{enumerate}
\section{Error}
Installation notes state that UCINET cannot be run on Mac. Need to use a Windows emulator to run UCINET. 

\section{User Story 8}
\subsection{Intention}
Download Wine for mac so that I can run UCINET on my Mac. 
\subsection{Action}
\begin{enumerate}
\item Download UCINET and save to Desktop. The url for this is: \href{https://sites.google.com/site/ucinetsoftware/downloads}{UCINET Software}
\item Install X11: Go to \href{https://www.xquartz.org}{XQuartz} and download the latest version. 
    \item Download: \href{http://winebottler.kronenberg.org}{WineBottler1.7.37} 
    \item Follow installation prompts then add Wine and WineBottle into Applications folder.
\end{enumerate}
To Run WineBottle:
\begin{enumerate}
    \item Find WineBottle in Applications
    \item Go to 'Advanced' and under 'Program Installation' click 'select file'
\end{enumerate}
\section{Error}
WineBottle unexpectedly closed and this error message appeared:\\
\includegraphics[width=12cm]{WineBottle.png}

\section{User Story 8}
\subsection{Intention}
Download Version 1.8.6 of WineBottle to see if it works.

\subsection{Action}
\begin{enumerate}
    \item Go to: \href{http://winebottler.kronenberg.org}{http://winebottler.kronenberg.org}
    \item Click 'Winebottler 1.8.6 Stable'
    \item Click 'Skip Ad' in top right corner 
    \item Once installed, add Wine and WineBottler to Applications
\end{enumerate}
Try to run WineBottler:
\begin{enumerate}
    \item Open spotlight - search WineBottler 
    \item Click the application and then click 'Advanced' 
    \item Under 'Program Installation' click 'selection file' and select 'Setup32UCI6685.exe' on Desktop
    \item Under 'Winetricks' search 'mdac28' and select it.
    \item Click install and follow prompts
    \item Once installation is finished it asks which file is my 'Startfile' - select 'Program Files/Analytic Technologies/Uci6.exe'
\end{enumerate}
\section{Error}
Still can't open NetDraw.

\section{User Story 8}
\subsection{Intention}
Try use to Paper Machines add-on instead. Link here: \href{http://papermachines.org/install/}{Paper Machines}

\subsection{Action}
\begin{enumerate}
    \item Click the link to install 
    \item Open Zotero
    \item Click tools - add-ons - install new add on from file 
    \item Click Paper Machine download
\end{enumerate}

\subsection{Error}
Got error message. Did not work. 

\section{User Story 8}
\section{Intention}
See if Zotero Voyant export tool can visualise connections and relationships. Note, VoyantServer allows you to handle large texts without the connection timing out. 

\section{Action}
Link: \href{http://docs.voyant-tools.org/resources/run-your-own/voyant-server/}{Voyant Export}
\begin{enumerate}
    \item Download Java
    \item Download VoyantServer.zip file 
    \item Run server by double clicking on download
    \item Voyant control panel will appear (see image below)
\end{enumerate}
\includegraphics[width=12cm]{voyantserver.png}

\begin{enumerate}
    \item VoyantServer will automatically launch in browser. 
    \item Right click on collection then click 'export collection to voyant'
    \item Save zip file to desktop as 'Bali.zip'
    \item Open Voyant and upload Bali.zip file to Voyant. 
\end{enumerate}
NOTE: List of Voyant tools: \href{https://voyant-tools.org/docs/#!/guide/tools}{Here}\\
\includegraphics[width=15cm]{voyant.png}

\subsection{Error}
None.

\section{Hypothes.is}
\subsection{Intention}
Text Hypothes.is to see if it is better than Voyant.
\subsection{Action}
\begin{itemize}
    \item Go to Hypothes.is website
    \item Create login
    \item Upload pdf 
    \item highlight pdf 
    \item annotate pdf
    \item save 
\end{itemize}
\subsection{Error}
No error but looks difficult to export. Exporting is not essential for PoC but would be useful. After doing some research apparently you can import annotations into OpenSemantic Desktop Search. 

\section{Following meeting with Brian}
\subsection{Intention}
Try to use OpenSemantics for searching texts and tagging. 
\subsection{Action}
\begin{enumerate}
    \item Download VirtualBox: \href{https://www.virtualbox.org}{https://www.virtualbox.org}
    \item Download OpenSemantics.org: \href{https://www.opensemanticsearch.org}{https://www.opensemanticsearch.org}
    \item Open VirtualBox 
    \item Click import 
    \item Select OpenSemantics file 
    \item Double click OpenSemantics 
\end{enumerate}
\subsection{Error}
When I tried to launch OpenSemantic I got a message saying that I needed a login and password. 

\section{OpenSemantic Desktop Search }
\subsection{Intention}
Attempt to fix previous error by re-downloading and following the manual again.
\subsection{Action}
\begin{enumerate}
    \item Open VirtualBox
    \item Download \verb|open-semantic-desktop-search_19.07.19.ova|: \href{https://www.opensemanticsearch.org}{https://www.opensemanticsearch.org}
    \item Start VirtualBox 
    \item Click 'file'
    \item Click 'import appliance'
    \item Choose the downloaded open-semantic-desktop-search file.
    \item Click settings 
    \item Click shared folders 
    \item select folder with relevant files
    \item double click OpenSemantic on left panel to start
\end{enumerate}
\subsection{Error}
No error but tool ran extremely slowly. I Googled it and apparently VirtualBox is slow.

\section{OpenSemantic Desktop Search}
\subsection{Intention}
Make sure that it can:
\begin{itemize}
    \item Tag sources
    \item Search Sources
    \item Import from Hypothes.is
\end{itemize}
\subsection{Action}
\begin{enumerate}
    \item Open VirtualBox
    \item Make sure shared folder is the folder that contains my sources
    \item Launch OpenSemantic 
    \item Click on annotations and tagging 
    \item add tag 
    \item save (successful)
    \item Go back to home page 
    \item Click search
    \item Search for word 'stress' (successful)
    \item Go to Datasource 
    \item Click Hypothesis
    \item Click import
    \item Go to Hypothes.is website \href{https://hypothes.is}{https://hypothes.is}
    \item Click Settings
    \item Click Developer
    \item Copy API token
    \item Go back to OpenSemantic
    \item Paste API token 
    \item Click import
\end{enumerate}
Note: more detailed steps are outlined in PoC Design.
\subsection{Error}
OpenSemantic seems to be able to successful search, tag and import from Hypothes.is. While there were no errors, the automatic tagging seemed to pick up the wrong words. Next time I will convert a pdf to .txt file to see if OpenSemantic is more efficient.

\end{document}
